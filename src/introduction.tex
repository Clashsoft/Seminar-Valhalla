\section{Einleitung}\label{sec:introduction}

Die Programmiersprache Java wurde erstmals im Jahr 1995 von Sun Microsystems vorgestellt.
Zu dieser Zeit gängige Hardware unterscheidet sich zu heutiger nicht nur in der Performance, sondern auch im Aufbau.
Speicherzugriffe und arithmetische Operationen hatten beispielsweise einen vergleichbaren Zeitaufwand und 64-Bit Systeme waren nur bei Servern und nicht für Endanwender vertreten.
Inzwischen hat sich die Hardware weiterentwickelt, während das Speichermodell der Sprache weitgehend unverändert blieb.
Das Projekt \emph{Valhalla} hat die Aufgabe, Java in diesem Aspekt zu modernisieren.
Um das zugrundeliegende Problem darzustellen, wird zunächst in Abschnitt~\ref{sec:memory-model} das derzeitige Speichermodell von Java erläutert und einige Grundbegriffe definiert.
In Abschnitt~\ref{sec:valhalla} folgt dann ein Überblick über das Projekt Valhalla.
Darin wird auf verschiedene Ansätze zur Modernisierung, die während des Projekts entstanden sind, sowie geplante Neuerungen für Java eingangen.
Zuletzt gibt Abschnitt~\ref{sec:summary} eine Zusammenfassung dieser Arbeit.
