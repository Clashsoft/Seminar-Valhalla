\section{Einleitung}\label{sec:introduction}

Als im Jahre 1995 die Programmiersprache Java von Sun Microsystems vorgestellt wurde, war heutige Hardware noch weit entfernt.
Speicherzugriffe und arithmetische Operationen hatten einen vergleichbaren Zeitaufwand und 64-Bit Systeme waren nur bei Servern nicht für Endanwender vertreten.
Inzwischen hat sich die Hardware weiterentwickelt, aber nicht die Sprache.
Während moderne Computer häufig größere Speicherzellen verarbeiten können und durch \ac{cpu}-Caches der Zugriff auf den Hauptspeicher vergleichsweise lange dauert, bietet Java lediglich 64-Bit Werte an und sorgt durch Indirektion im Objektmodell für lange Zugriffszeiten.
Aus diesem Grund soll nun die Sprache weiterentwickelt werden, indem eine Alternative zu Referenz-basierten Objekten erschaffen wird.
Dies ist Aufgabe des Projekts \emph{Valhalla}, welches in Abschnitt~\ref{sec:valhalla} näher untersucht wird.
