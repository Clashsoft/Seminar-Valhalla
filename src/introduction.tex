\section{Einleitung}\label{sec:introduction}

Als im Jahre 1995 die Programmiersprache Java von Sun Microsystems vorgestellt wurde, war heutige Hardware noch weit entfernt.
Speicherzugriffe und arithmetische Operationen hatten einen vergleichbaren Zeitaufwand und 64-Bit Systeme waren nur bei Servern und nicht für Endanwender vertreten.
Inzwischen hat sich die Hardware weiterentwickelt, während das Speichermodell der Sprache weitgehend unverändert blieb.
Das Projekt \emph{Valhalla} hat die Aufgabe, Java in diesem Aspekt zu modernisieren.
Um das zugrundeliegende Probleme darzustellen, wird zunächst in Abschnitt~\ref{sec:memory-model} das derzeitige Speichermodell von Java erläutert und einige Grundbegriffe definiert.
In Abschnitt~\ref{sec:valhalla} folgt dann ein Überblick über das Projekt Valhalla.
Darin wird auf die Geschichte des Projekts sowie damit verbundene Neuerungen für Java eingangen.
Zuletzt gibt Abschnitt~\ref{sec:summary} eine Zusammenfassung und einen Ausblick.
