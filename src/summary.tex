\section{Zusammenfassung}\label{sec:summary}

% Zusammenfassung
Aus dem Wunsch, die Sprache Java auf das Speichermodell von heutiger Hardware zu modernisieren, hat sich mit Valhalla ein großes und aufwendiges Projekt entwickelt.
Die Value Types erfordern zwar nur kleine syntaktische Änderungen, dafür jedoch weitreichende Änderungen der \ac{jvm}.
Mehr sprachliche Neuerungen und Probleme entstehen jedoch durch die generische Spezialisierung, die nicht nur für primitive Datentypen interessant ist, sondern auch große Bedeutung für den Erfolg von Value Types hat.

Seit die Entwicklung des Projekts im Jahre 2014 begann, wurden viele Lösungansätze untersucht und in Form verschiedener Prototypen umgesetzt.
Da stets die Kompatibilität im Vordergrund stand, mussten die weitreichenden Änderungen genau abgewägt und analysiert werden.
Im Jahr 2020 ist mit den Inline-Klassen ein vielversprechendes Konzept vorhanden, das bei ersten Tests zwar gut funktionierte, aber noch nicht alle Versprechen erfüllen konnte.
Daher ist es naheliegend, dass die Entwicklung der Value Types noch weitere Zeit in Anspruch nehmen wird.

Auch auf dem Weg zur generischen Spezialisierung sind viele Prototypen entstanden, die zum Teil einfach umzusetzen waren, aber unterschiedlich viele Probleme und Bedenken mit sich führten.
Der neueste Ansatz der Spezialisierung zur Laufzeit verlangt zwar einen hohen Implementierungsaufwand seitens der \ac{jvm}-Autoren, bietet aber bessere Optimierungsmöglichkeiten als andere Ansätze.
% Ausblick
Insgesamt ist die Implementierung der generischen Spezialisierung stark von der Umsetzung der Value Types abhängig.
Da beide Projekt noch unvollständig umgesetzt sind, ist derzeit nicht absehbar, wann die beiden Features in einer Java-Version veröffentlich werden.
