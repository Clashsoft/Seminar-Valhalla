\section{Zusammenfassung}\label{sec:summary}

Das OpenJDK-Projekt Valhalla hat das Ziel, die Sprache Java und die \ac{jvm} dahingehend anzupassen, dass moderne Hardware besser genutzt werden kann.
Es strebt eine Alternative für Referenz-basierte Objekte an, um Indirektion zu vermeiden und damit sowohl Zugriffszeiten als auch Speicherverbrauch zu verringern.
Nachdem Java-Entwickler mit Value-basierten Klassen bereits in Java 8 an ein solches Konzept herangeführt wurden, stellt das Projekt nun sprachliche Unterstützung mit den Inline-Klassen bereit.
Während diese wie gewöhnliche Klassen programmiert werden, verhalten sie sich in vielen Aspekten wie primitive Typen.
Dabei wird Indirektion vermieden, wodurch höhere Effizienz zur Laufzeit erreicht werden kann.

Ein weiteres Problem, das mit Projekt Valhalla behoben werden soll, betrifft die generischen Typen in Java.
Diese sind derzeit nicht kompatibel mit primitiven Typen, da sie mit dem Ansatz der Erasure implementiert wurden.
Mit der Einführung der Inline-Klassen wird das Problem verstärkt, da ihre Vorteile in generischen Datenstrukturen mit Erasure nicht anwendbar sind.
Abhilfe soll die Spezialisierung schaffen, die durch die Erweiterung von Compiler und Laufzeit implementiert wird und die Probleme der Erasure lösen soll.
Es hat sich im Verlauf des Projekts Valhalla erwiesen, dass diese Kombination am besten geeignet ist, um die Spezialisierung effizient und abwärtskompatibel einzuführen.
Aufgrund des damit verbundenen Implementierungsaufwands und bisher ungelösten Problemen ist das Projekt jedoch noch nicht vollendet.
