\section{Projekt Valhalla}\label{sec:valhalla}

Das OpenJDK-Projekt \emph{Valhalla}~\cite{openjdk-valhalla} besteht aus drei wesentlichen Neuerungen, die in folgenden \acp{jep} definiert sind:

\begin{itemize}
    \itemsep0em
    \item \ac{jep} 169: Value Objects~\cite{jep-169}
    \item \ac{jep} 193: Variable Handles~\cite{jep-193}
    \item \ac{jep} 218: Generics over Primitive Types~\cite{jep-218}
\end{itemize}

Grundlage für das Projekt stellen die \emph{Value Objects} dar.
Das Ziel des \ac{jep} 169 ist, die \ac{jvm} dahingehend zu erweitern, dass Objekte zur Laufzeit als Werte repräsentiert werden können.
Es ist jedoch explizit nicht vorgesehen, Änderungen an der Sprache durchzuführen.
Bei den Value Objects handelt es sich also um ein reines \ac{jvm}-Feature, das für Programmierer zunächst nicht direkt zugänglich ist.
Im Projekt Valhalla hingegen wird auch sprachliche Unterstützung für Objekte als Werte angestrebt.
Dies wird in Unterabschnitt~\ref{subsec:value-types} betrachtet.

Die \emph{Variable Handles} bezeichnen einen Teil der Java-Standardbibliothek, die präzisere Zugriffskontrolle auf Felder von Objekten und Array-Elementen im Kontext von nebenläufigen Programmen ermöglichen.
Die Implementierung der \ac{api} erfolgte bereits in Java 9 und wird daher nicht näher in dieser Arbeit betrachtet.

Die Erweiterung von generischen Typen um Unterstützung für primitive Typen ist der größte Teil des Projekts.
Sie verlangt weitreichende Änderungen, die sowohl die Sprache als auch die \ac{jvm} betreffen.
In Unterabschnitt~\ref{subsec:primitive-generics} erfolgt eine Vertiefung in dieses Thema.

\subsection{Value Types}\label{subsec:value-types}

\emph{Value Types} bezeichnen ein Sprachfeature, das benutzerdefinierte Typen erlauben soll, die keine Referenzsemantik besitzen.
Das bedeutet, dass Instanzen dieser Typen als reine Werte repräsentiert werden und keine Referenzen verwendet werden.
Wie bereits am Ende von Abschnitt~\ref{sec:memory-model} beschrieben, sollen diese aus Sicht des Programmierers wie Klassen funktionieren, sich aber wie primitive Datentypen verhalten.
In diesem Abschnitt soll zunächst betrachtet werden, wie sich Value Types von Referenztypen unterscheiden.
Daraufhin wird ein Vorgänger der Value Types vorgestellt, die \emph{Value-basierten Klassen}.
Diese bilden die Grundlage für die eigentliche Implementierung der Value Types, die \emph{Inline-Klassen}.

\subsubsection{Unterschiede zu Referenztypen}

Es soll nun betrachtet werden, was Value Types von regulären Referenztypen unterscheided.
Größter Unterschied ist, dass nun keine Pointer mehr verwendet werden, ohne die sich einige grundlegende Konzepte ändern.
So besitzen Werte keine Identität mehr, die verglichen werden könnte, und damit auch keinen Identitäts-Hashcode.
Die Veränderung der Felder von Werten ist entweder nicht erlaubt oder wirkt sich nur auf lokale Instanzen in Variablen und Parametern aus, da Value Types wie primitive Werte bei jeder Zuweisung kopiert werden.
Durch die Kopier-Semantik können Werte nicht mehr als Monitor für die Synchronisierung dienen und auch die \code{clone}-Operation ist nicht mehr sinnvoll.
Ebenso ist der Finalizer nun überflüssig, da die Werte nicht mehr vom Garbage Collector verarbeitet werden.
Auch \code{null} ist kein gültiger Wert mehr, allerdings wäre eine Alternative möglich, die den Speicher für den Wert auf null setzt.
Zuletzt kann nicht von Value Types geerbt werden, da im Voraus bekannt sein muss, wie viel Speicherplatz reserviert werden muss, was mit Vererbung nicht statisch entschieden werden kann.

\subsubsection{Value-basierte Klassen}

Seit Java 8 gibt es bereits ein Konzept, das den Value Types ähnelt: die Value-basierten Klassen~\cite{value-based-classes}.
Dabei handelt es sich um eine Konvention, die von einigen Klassen der Standardbibliothek eingehalten wird und auch für Bibliotheksautoren und andere Endbenutzer der Sprache verfügbar ist.
Value-basierte Klassen setzen voraus, dass alle Felder \code{final} sind, wodurch Instanzen dieser Klassen unveränderbar sind.
Weiterhin dürfen diese Klassen keinen öffentlich zugänglichen Konstruktor besitzen, sondern müssen stets über eine Factory-Methode instanziiert werden.
Anders als ein Konstruktor, der beim Aufruf immer eine neue Instanz erstellt, darf eine Factory-Methode auch das selbe Objekte mehrfach zurückgeben, beispielsweise wenn sie ein Cache verwaltet.
Da der Benutzer einer Value-basierten Klasse keine Kontrolle über die Identität der Instanzen hat, ist es nicht sinnvoll, Identitätsvergleiche mit \code{==} durchzuführen.
Folglich müssen stets \code{equals} und \code{hashCode} implementiert werden.
Identitäts-basierte Operationen und Synchronisierung auf Value-basierten Klassen sind zwar nicht verboten, da es sich um eine Konvention handelt, können aber zu unerwarteten Ergebnissen führen.
Somit erfüllen Value-basierte Klassen fast alle zuvor genannten Punkte, durch die sich Value Types von Referenztypen unterscheiden.
Ausnahmen sind \code{null} als möglicher Wert sowie die erlaubte Vererbung.
Beides ist jedoch bei derzeitigen Beispielen für Value-basierte Klassen wie \code{Integer} und \code{Optional} ohnehin nicht üblich.

\subsubsection{Inline-Klassen}

Mit dem Projekt Valhalla soll die Konvention der Value-basierten Klassen nun Sprach- und \ac{jvm}-Unterstützung erhalten.
Diese trägt die Bezeichnung \emph{Inline-Klassen}~\cite{object-model}, die sich aus dem Schlüsselwort \code{inline} ableitet, das für deren Deklaration eingesetzt wird.
Inline-Klassen sollen als vollwertige Implementierung der Value Types dienen.
Listing~\ref{inline-class-example} zeigt, wie eine Inline-Klasse aussehen kann.
Dabei ist zu beachten, dass sich der Code nur durch das Schlüsselwort \code{inline} von einer regulären Klasse unterscheidet.
Nach dem Motto ``Codes like a class, works like an int''~\cite{object-model} sollen die syntaktischen Änderungen bewusst klein sein.

\begin{listing}
    \begin{minted}{java}
        inline class Point {
            private int x, y;

            public Point(int x, int y) { this.x = x; this.y = y; }

            public int x() { return x; }
            public int y() { return y; }
        }
    \end{minted}
    \vspace{-3ex}
    \caption{Beispiel für eine Inline-Klasse (angepasst aus~\cite{object-model})}
    \label{inline-class-example}
\end{listing}

% Primitive untergeordnet
Mit der Einführung von Inline-Klassen wird die Unterscheidung zwischen primitiven und Referenz-Typen abgeschafft.
Nun wird zwischen Inline- und Referenz-Typen unterschieden;
die primitiven Typen werden den Inline-Typen untergeordnet.
% T.default
Weiterhin kann nun für alle Typen ein Default-Wert definiert werden, der mit \code{T.default} bezeichnet wird.
Für primitive Typen bleibt dieser \code{0} oder \code{false}, während Referenztypen \code{null} beibehalten.
Inline-Typen definieren ihren Default-Wert als eine Instanz, bei der alle Felder mit dem jeweiligen Default-Wert belegt sind.
Im Bezug auf das obige Beispiel ist der Wert \code{Point.default} folglich äquivalent zu \code{new Point(0, 0)}.

% Marker interfaces
Unter Umständen kann es nützlich sein, zur Laufzeit zu ermitteln, ob es sich bei einem Wert um eine Referenz oder einen Inline-Wert handelt.
Dafür werden neue Interfaces eingeführt, von denen automatisch und implizit alle Klassen erben:
\code{IdentityObject} für Referenzen und \code{InlineObject} für Inline-Klassen.
So kann entschieden werden, ob Identitätsoperationen wie \code{==} oder Synchronisierung sinnvoll sind.
% Interfaces + Klassenvererbung
Inline-Klassen dürfen auch reguläre Interfaces implementieren.
Ebenfalls können sie von Klassen ohne \code{inline} erben, solange diese nicht den Anforderungen an Inline-Klassen widersprechen.
Nicht erlaubt wären beispielsweise Superklassen mit nicht-finalen Feldern oder Methoden, die \code{synchronized} sind.
Ein Beispiel für eine Klasse, welche die Anforderungen erfüllt, ist \code{Object}, von der weiterhin alle Klassen erben.

% Boxing
Mit der Unterordnung von primitiven Typen werden auch diese ohne Boxing von \code{Object} erben.
Boxing wird dann zu einer Operation, die von der \ac{jvm} intern durchgeführt wird und nicht mehr nur durch den Compiler bereitgestellter syntaktischer Zucker ist.
% Arrays
Das hat zur Folge, dass Array-Typen anders voneinander erben.
Insbesondere kann nun ein \code{int[]} direkt in ein \code{Integer[]} oder ein \code{Object[]} umgewandelt werden.
Ein ähnliches Verhalten zeigen die benutzerdefinierten Inline-Klassen.
% Referenz-Wrapper T.ref
Anders als bei primitiven Typen gibt es jedoch dafür keine explizit definierten Referenz-Wrapper.
Stattdessen soll neue Syntax hinzugefügt werden, mit der Referenz-Wrapper für Inline-Klassen benannt werden können.
Diese nimmt die Form \code{T.ref} an, wobei \code{T} eine Inline-Klasse bezeichnet.
Im Gegensatz dazu steht die Form \code{T.val}, die explizit den Inline-Typ benennt.
Diese Formen sollen mit den bestehenden Namen für primitive und Wrapper-Typen kompatibel sein -- \code{int.ref} bezeichnet dann den selben Typ wie \code{Integer} und \code{Integer.val} entspricht \code{int}.

\subsection{Primitive und generische Typen}\label{subsec:primitive-generics}

Generische Typen sind ein Feature, das bereits in Java 5 eingeführt wurde.
Sie ermöglichen es, Klassen und Methoden zu definieren, die über verwendete Typen abstrahieren.
Dabei kommen Typparameter als Platzhalter zum Einsatz.
So kann beispielsweise eine \code{Map}-Klasse erstellt werden, bei der Schlüssel und Werte beliebig, aber fest typisiert sind.
Bei der Verwendung der Klasse werden die Typen festgelegt: \code{Map<String, Integer>}.
Dadurch können typsichere Datenstrukturen definiert werden.
In diesem Abschnitt wird betrachtet, wie generische Typen implementiert werden können und wie sich die verschiedenen Ansätze auf die Kompatibilität mit primitiven Typen auswirken.

\subsubsection{Erasure}

Die Implementierung von generischen Typen erfolgte durch die sogenannte \emph{Erasure}.
Beim Generieren von Bytecode ersetzt der Java-Compiler sämtliche Vorkommen von generischen Typparametern durch \code{Object}\footnote{Falls durch die Form \code{<T extends Bound>} ein Bound angegeben wird, wird der Typparameter durch diesen statt \code{Object} ersetzt.}.
Generische Typen konnten so implementiert werden, ohne Änderungen an der \ac{jvm} vornehmen zu müssen.

% Migration mit Erasure
Erasure ermöglichte ferner den Entwicklern, schrittweise auf generische Typen umsteigen.
Die Benutzung von \code{Map} war weiterhin ohne Angabe der Schlüssel- und Wert-Typen erlaubt -- dies wird als Raw Type bezeichnet und verhält sich ähnlich wie \code{Map<Object, Object>}.
Die Wahrung der Kompatibilität war ein besonderes Ziel bei der Entwicklung der generischen Typen.
Es sollte vermieden werden, dass Entwickler sämtlichen Code neu kompilieren müssen.

% Keine generischen Typen mit Primitives
Eine Limitierung der Erasure ist, dass primitive Typen nicht als Typparameter verwendet werden können.
So erzeugt der Compiler derzeit einen Error, wenn der Typ \code{List<int>} verwendet wird.
Das liegt daran, dass der Typ \code{int} derzeit nicht kompatibel mit \code{Object} ist.
Nur durch Boxing kann die Umwandlung \code{int <-> Integer <-> Object} geschehen.
Dies kann aber in generischen Klassen nicht sichergestellt werden, solange der konkrete Typ nicht bekannt ist.
Folglich muss statt \code{List<int>} eine \code{List<Integer>} verwendet werden.
Zur Laufzeit kann dies jedoch aufgrund des Boxings die Performance beeinträchtigen:
Sowohl das Boxing (Anlegen von \code{Integer}-Objekten) als auch das Unboxing (Extrahieren des \code{int}-Werts) benötigt Zeit, besonders aufgrund des Allokations- und Garbage-Collector-Overheads sowie der zusätzlichen Indirektion.
Der erhöhte Speicherverbrauch, der bereits im vorherigen Unterabschnitt~\ref{subsec:value-types} gezeigt wurde, ist auch hier in ähnlichem Umfang vorhanden.

\subsubsection{Händische Spezialisierung}

Aufgrund von Bedenken zu Performance und Speicherverbrauch sind in Java-Bibliotheken häufig händische Spezialisierungen vertreten.
Beispiele dafür sind selbst in der Standardbibliothek zu finden.
So gibt es seit Java 8 die Klassen \code{OptionalInt} und \code{IntStream} sowie Äquivalente für \code{long} und \code{double}, die als performanter und speichersparender Ersatz für \code{Optional<T>} und \code{Stream<T>} dienen und untereinander nicht kompatibel sind~\cite{java-8-docs}.
Das Problem wird verstärkt, wenn die zu spezialisierende Klasse mehr als einen Typparameter hat.
Im Package \code{java.util.function} ist eine Reihe von Spezialisierungen für Interfaces vorhanden, die je eine Methode mit null, einem oder zwei Parametern bereitstellen, deren Typen generisch sind.
Mehrere Klassen variieren ebenfalls im Rückgabetyp der Methode.
Händische Spezialisierungen können zwar für verbesserte Performance sorgen, besonders wenn typspezifische Optimierungen vorgenommen werden, erzeugen aber einen erhöhten Wartungsaufwand und sind fehleranfälliger.

% Einfluss von Value Types
In bisherigen Java-Versionen ist das Problem nur auf die acht primitiven Typen beschränkt.
Werden diese acht auf die drei häufigsten -- \code{int}, \code{long} und \code{double} -- reduziert, kann die Anzahl der händischen Spezialisierungen eingeschränkt werden.
Dies ist jedoch nicht mehr möglich, wenn mit Value Types eine unendliche Zahl neuer Typen hinzukommen, die sich ähnlich wie primitive verhalten sollen.
Es wäre nicht sinnvoll, den Speicherverbrauch von Arrays zu verringern, wenn das nicht auch mit Listen möglich ist, da diese in den meisten Fällen die bessere Wahl sind.
Folglich ist es naheliegend, dass die Änderung von generischen Typen zusammen mit den neuen Value Types eingeführt wird.

\subsubsection{Spezialisierung durch den Compiler}

Ein Ansatz, der das Wartungsproblem der händischen Spezialisierung vermeidet, ist Spezialisierung durch den Compiler.
Diese Technik kommt beispielsweise in C++ zum Einsatz, wo der Compiler Typparameter wie bei einer Vorlage durch die konkreten Typen ersetzt, als wären sie direkt in der Klasse oder Methode verwendet worden.
Das hat den Vorteil, dass typspezifische Optimierungen durchgeführt werden können und Spezialisierungen eigene Methoden und sonstige Funktionalität bereitstellen können.
In ähnlicher Form wird der Ansatz auch von der \ac{jvm}-Sprache \emph{Scala} verwendet, in der Spezialisierungen mit einer Annotation an einem Typparameter festgelegt werden können~\cite{scala-specialized}.
Das verringert zwar den Programmieraufwand, sorgt aber dennoch für größere Artefakte aufgrund des duplizierten Bytecodes.
Zudem sind die entstehenden Klassen aus Sicht der \ac{jvm} und anderen Java-ähnlichen Sprachen nicht verwandt, wodurch spezielle Interfaces verwendet werden müssen, wenn beispielsweise ein Wildcard-Typ gebraucht wird.
Weiterhin wird Reflection erschwert, was in Scala durch eine eigene Reflection-Bibliothek umgangen wird.
Eine solche Vorgehensweise ist in zukünftigen Java-Versionen aufgrund von Kompatibilitätsbedenken also nicht erstrebenswert.

\subsubsection{Spezialisierung zur Laufzeit}

Eine kompatible und nachhaltige Implementierung von generischer Spezialisierung erfordert folglich Unterstützung durch die Laufzeitumgebung.
Bereits im Jahr 2014 wurde an einem ersten Prototyp gearbeitet, der Teile der Spezialisierung auf die Laufzeit auslagert~\cite{specialization}.
Der Java-Compiler erzeugt dann nur eine Bytecode-Datei pro Klasse, von der ausgehend Spezialisierungen zur Laufzeit erzeugt werden, wenn sie benötigt werden.
Dafür mussten Metadaten im Bytecode-Format hinzugefügt werden, die Angaben über verwendete Typparameter machen, die bei der reinen Erasure zuvor nicht notwendig waren.
Ein wichtiges Ziel war es, die Spezialisierung zur Laufzeit so einfach wie möglich zu halten und zusätzliche Verifikation oder komplexe Übersetzungsverfahren zu vermeiden.
Das sollte die Implementierung vereinfachen und Ladezeiten gering halten, setzte aber bedachte Änderung des Bytecode-Formats voraus.

\subsubsection{Syntaktische Neuerungen}

% any
Zum aktuellen Zeitpunkt ist noch keine endgültige Syntax für generischen Typen mit Spezialisierung festgelegt.
Es ist naheliegend, dass aus Kompatibilitätsgründen die bestehende Syntax für Typparameter die Semantik der Erasure beibehalten wird und Spezialisierung explizit aktiviert werden muss.
Dafür wurde in der Literatur zu Valhalla vorläufig das Schlüsselwort \code{any} verwendet: \code{class Box<any T>}.

% Layers
Eine weitere vorgeschlagene syntaktische Neuerung sind die \emph{Layers}, welche es ermöglichen, in bestimmten Spezialisierungen neue Methoden hinzuzufügen.
Sie sind aus dem Problem entstanden, dass die meisten bestehenden generischen Klassen unter der Annahme implementiert wurden, dass sie nur mit Referenztypen verwendet werden.
So kann beispielsweise die Methode \code{V get(K key)} der Klasse \code{Map<K, V>} \code{null} zurückgeben, wenn der Schlüssel nicht vorhanden ist.
Dies ist jedoch nicht möglich, wenn \code{V=int} ist.
Mithilfe von Layers kann nun festgelegt werden, dass es die \code{get}-Methode nur dann geben soll, wenn \code{V} ein Referenztyp ist.
Andernfalls könnte eine neue Methode \code{getOrDefault} hinzugefügt werden.
Listing~\ref{lst:map-layer} zeigt, wie dies im Code definiert werden könnte.
Das Schlüsselwort \code{layer} gibt hier die Methoden an, die nur in der Spezialisierung für Referenztypen \code{V} existieren sollen.
Ähnliche Syntax könnte eingesetzt werden, um gänzlich neue Methoden wie \code{List<int>.sum()} hinzuzufügen.

\begin{listing}
    \begin{minted}{java}
        interface Map<any K, any V> {
            V getOrDefault(K key, V defaultValue);

            layer<ref V> {
                V get(K key);

                default V getOrDefault(K key, V defaultValue) {
                    return containsKey(key) ? get(key) : defaultValue;
                }
            }
        }
    \end{minted}
    \vspace{-3ex}
    \caption{Beispieldefinition des Map-Interfaces mit Layers (aus~\cite{specialization}).}
    \label{lst:map-layer}
\end{listing}

% Weitere Probleme
Layer sind eine Lösung für eines von vielen konzeptionellen und implementierungsbezogenen Problemen mit der Spezialisierung.
Weitere Hindernisse bei Wildcard-Typen, Arrays, Reflection und anderen Bereichen der Sprache sind in~\cite{jep-218} und~\cite{specialization} näher erläutert.
