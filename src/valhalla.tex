\section{Projekt Valhalla}\label{sec:valhalla}

Das OpenJDK-Projekt \emph{Valhalla}~\cite{openjdk-valhalla} besteht aus drei wesentlichen Neuerungen, die in folgenden \acp{jep} definiert sind:

\begin{itemize}
    \item \ac{jep} 169: Value Objects~\cite{jep-169}
    \item \ac{jep} 193: Variable Handles~\cite{jep-193}
    \item \ac{jep} 218: Generics over Primitive Types~\cite{jep-218}
\end{itemize}

Grundlage für die Änderungen des Projekts stellen die \emph{Value Objects} dar.
Deren Ziel ist es, die \ac{jvm} mit Unterstützung für neue benutzerdefinierbare Typen zu erweitern, die im Vergleich zu bestehenden Klassen keine Referenzsemantik haben, sondern sich wie primitive Datentypen verhalten.
Unterabschnitt~\ref{subsec:value-types} erläutert diese näher.

Die \emph{Variable Handles} bezeichnen einen Teil der Java Standard-Bibliothek, die präziseren Zugriff auf Felder von Objekten und Array-Elementen ermöglichen.
Sie erlaubt unter Anderem die Kontrolle der Zugriffsreihenfolge durch mehrere Threads. % TODO can be removed
Die Implementierung der API erfolgte bereits in Java 9 und wird daher nicht näher in dieser Arbeit betrachtet.

Die Erweiterung von generischen Typen mit Unterstützung für primitive Typen ist der größte Teil des Projekts.
Sie verlangt weitreichende Änderungen, die sowohl die Sprache als auch die \ac{jvm} betreffen.
In Unterabschnitt~\ref{subsec:primitive-generics} erfolgt eine Vertiefung in dieses Thema.

\subsection{Value Types}\label{subsec:value-types}

\subsection{Primitive und generische Typen}\label{subsec:primitive-generics}
